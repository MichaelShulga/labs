\documentclass[a4paper, 12pt]{article}
\usepackage[a4paper,top=1.5cm, bottom=1.5cm, left=1cm, right=1cm]{geometry}
\usepackage[utf8]{inputenc}
\usepackage{mathtext}
\usepackage{amsmath}
\usepackage{amsfonts}
\usepackage[english, russian]{babel}
\usepackage{indentfirst}
\usepackage{longtable}
\usepackage{graphicx}
\graphicspath{{pictures/}}
\DeclareGraphicsExtensions{.pdf,.png,.jpg}
\usepackage{natbib}
\usepackage{svg}

\title{Лабораторная работа 1.1.3 Статистическая обработка результатов многократных измерений}
\author{Шульга Михаил}
\date{Сентябрь 2023}

\begin{document}
\maketitle

\section{Аннотация}

\subsection*{Цель работы:}
Применение методов обработки экспериментальных данных при измерении сопротивлений. Использование статистических методов. 

\subsection*{В работе используются:}
Набор резисторов (270 штук), универсальный цифровой вольтметр, работающий в режиме "Измерение сопротивлений постоянному току".

\section{Теоретические сведения}

    \subsection*{Среднее значение сопротивления:}
    $$\langle R\rangle = \frac{1}{N} \sum_{i = 1}^{N} R_i$$
    $N - число\: резисторов$\\
    $R_i - значение\:сопротивления$

    \subsection*{Интервал изменения сопротивления:}
    $$ \Delta R = \frac{R_{макс}-R_{мин}}{m}$$
    $R_{мин} - минимальное\: сопротивление$\\
    $R_{макс} - максимальное\: сопротивление$\\
    $m - количество\: частей$

    \subsection*{Построение гистограммы:}
    Гистограмму будем строить следующим образом. По оси абсцисс откладываем сопротивление резистора и отмечаем интервалы изменения сопротивления. А по оси ординат над каждым интервалом можно откладывать число результатов $\Delta n$, которое попадает в данный интервал. Удобнее будет это число разделить на число всех измерений и на ширину используемого интервала $\Delta R$.
    $$y = \frac{\Delta n}{N\Delta R}$$

    На том же графике отложим по оси абсцисс среднее значение сопротивления.
    
    Для характеристики разброса случайной величины используется среднеквадратичное отклонение
    $$ \sigma = \sqrt{\frac{1}{N} \sum_{i = 1}^{N}(R_i - \langle R\rangle)^2}$$
    
    На оси абсцисс полезно будет отметить точки $\langle R\rangle-\sigma$ и $\langle R\rangle+\sigma$, чтобы посмотреть, как располагается гистограмма относительно этих точек.
    \par
    Используя $\sigma$, можно построить функцию распределения Гаусса
    $$y=\frac{1}{\sqrt{2\pi}\sigma}e^{-\frac{(R-\langle R\rangle)^2}{2\sigma^2}}$$
    
    Эту зависимость нанесём на гистограмму.

\section{Методика измерений}

Подготовить вольтметр к работе, включив его в сеть и дав ему прогреться в течение 15-20 минут.

Провести измерения сопротивлений резисторов изнабора N = 250-300 штук.



\section{Используемое оборудование}

\subsection*{Обоснование пренебрежения погрешностью при измерении}
    Поскольку наш цифровой вольтметр обеспечивает точность до сотых долей процента относительной погрешности, погрешностью измерений, связанных с ним можно пренебречь по сравнению с отклонениями от номинала, полученными в процессе изготовления резисторов.

\section{Результаты измерений и обработка данных}

\subsection*{Результаты измерения сопротивления 270 резисторов}

\input{{table.tex}}

\subsection*{Построение графиков}

Построим по 2 графика $m = 10$ и $m = 20$ для
своих измерений(Шульга),
измерений подгруппы(Шуминов, Шульга, Хабибуллин),
измерений всей группы, предварительно вычислив $\langle R\rangle$ и $\sigma$.

\begin{figure}[h!]
    \centering
    \includesvg[width = 1\linewidth ]{resistorsShulga.svg}
    \caption{Графики своих измерений(Шульга)}
    \label{graph1}
\end{figure}

\begin{figure}[h!]
    \centering
    \includesvg[width = 1\linewidth ]{resistorsSSH.svg}
    \caption{Графики измерений подгруппы(Шуминов, Шульга, Хабибуллин)}
    \label{graph2}
\end{figure}

\begin{figure}[h!]
    \centering
    \includesvg[width = 1\linewidth ]{resistorsSSH.svg}
    \caption{Графики измерений всей группы}
    \label{graph3}
\end{figure}



\section{Обсуждение результатов}

\subsection*{Сравнение $\langle R\rangle$ с номиналом}

Для измерений подгруппы $\langle R\rangle = 499,8 \:Ом$. Считая 500 Ом номиналом, получаем, что погрешность составляет 0,64\%.

\subsection*{Соотвествие гистограммы и распрделения Гаусса}

Видно, что Гауссово распрделение соотвествует гистограмме. При увелечении количества измерений соотвествие становится выше. Но при наименьшем количестве измерений (Рис. 1) и сильном разбиении $m = 20$, видны серьезные несоотвествия.


\subsection*{Вероятность попадания в определенный интервал}

В интервал от $\langle R\rangle - \sigma$ до $\langle R\rangle + \sigma$ укладывается 85$\%$ значений, в интервал от $\langle R\rangle - 2\sigma$ до $\langle R\rangle + 2\sigma$ укладывается 97$\%$ значений.

Практически мы получаем, что величина сопротивления резистора, наугад выбранного из данного набора, попадает в интервал $499,80 \pm 1,35 \:Ом$ с вероятностью 46\%, в интервал $499,80 \pm 18\: Ом$ - с вероятностью 97\%.

\section{Заключение}

В данной лабораторной работе были проведены измерения сопротивлений 270 резисторов и осуществлена их статистическая обработка. Мы вычислили среднее значение сопротивления, построили гистограммы и сравнили результаты с нормальным распределением. Работа позволила понять, какие значения сопротивлений наиболее вероятны, и как они соотносятся с ожидаемыми значениями.

\end{document}
